\section{System Disk Encryption}

An encrypted system disk, prevents that the data contained in it can be clone or replicated without the passphrase or authentication server. For this design, the disks will be encrypted through kickstart passphrase and then removed once the remote Tang server are reached, which means, of non-authorized user gains physical access to the server:
\begin{itemize}
  \item Once the server boots, it will require root credentials.
  \item If halted and attempt to change the root password, the encryption passphrase prompt will be requested - which was deleted.
  \item If booted through a Live USB OS, the encrypted partitions remain unreadable.
  \item If the drive is taken, the disk would never gain access to its content.
\end{itemize}

\newpage
\subsection{LUKS - Linux Unified Key Setup}

According to a paper subscribed by Danut Anton and Emil Simion \footnote[1]{https://ieeexplore-ieee-org.usm.idm.oclc.org/stamp/stamp.jsp?tp=\&arnumber=8678978}, LUKS is one of the most common FDE solution for Linux based systems.
FDE works by encrypting every single bit on a storage device, so if the user doesn't have the password, data cannot be recovered. The most common problem for FDE solutions is password management, which at what concerns this implementation, will be handled by a two level key hierarchy. A strong master key is generated by an OS, which is used for encrypt/decrypt the hard drive. That key has to be split and encrypted with secret user key and stored on the device, at the beginning of the memory. The advantage of this approach is that you can have multiple systems with multiple keys, allowing you to have multiple decryption Servers.

\vskip 2cm
\begin{figure}
  \includegraphics[width=14cm]{images/image2.png}
  \centering
  \caption{LUKS Operational Diagram}
\end{figure}

\newpage
\subsection{Clevis}

Clevis is a pluggable framework for automated decryption. It can be used to provide automate decryption of data or even automated unlocking of LUKS volumes \footnote[2]{https://github.com/latchset/clevis}.
To write: Clevis can subscribe up to 8 keys to 8 different servers/users and it can be restricted to how many of them are required as minimum; by default 1.


\subsection{Clevis Puppet Profile and Role}


\newpage
\section{Tang Server - Decryption Agent}

\subsection{Tang Puppet Profile and Role}

\newpage
\section{Lab Testing - Proof of Concept}
\subsection{Kickstart Modifications - Use of Encryption in Provisioning Template}
Since the drive must me encrypted with LUKS early during the provisioning, new Kickstart Provisioning Template and Partition Tables had to be created at Foreman.

\vskip 0.5cm
\begin{lstlisting}[language=bash]
#Encrypted VDA - Partition Table

ignoredisk --only-use=${BOOT_DEV}
zerombr
clearpart --drives=${BOOT_DEV} --all --initlabel
part /boot     --size=1024 --asprimary --ondrive=${BOOT_DEV}
part /boot/efi --size=200  --asprimary --ondrive=${BOOT_DEV} --fstype=efi
part pv.boot   --size=1 --grow  --encrypted --passphrase=temppass --ondisk=${BOOT_DEV}
volgroup ${BOOT_VG} pv.boot
logvol /               --vgname=${BOOT_VG} --size=1 --grow --name=root
\end{lstlisting}

"Encrypted VDA" initialize the System disk with two regular partitions - /boot and /boot/efi - and then a PV, a VG and a LV, been the LV encrypted through LUKS with a temporary password
\vskip 0.5cm
\begin{lstlisting}[language=bash]
##Kickstart - Encrypted Provisioning Template
#Packages Section
%packages
clevis-dracut
#Post Section
%post --log=/mnt/sysimage/root/install.post.log
curl -sfg http://tang01.cp.lsst.org/adv -o adv1.jws
clevis luks bind -f -k- -d /dev/vda3 \
tang '{"url":"http://tang01.cp.lsst.org","adv":"adv1.jws"}' <<< "temppass"
curl -sfg http://tang02.cp.lsst.org/adv -o adv2.jws
clevis luks bind -f -k- -d /dev/vda3 \
tang '{"url":"http://tang02.cp.lsst.org","adv":"adv2.jws"}' \ <<< "temppass"
cryptsetup luksRemoveKey /dev/vda3 <<< "temppass"
\end{lstlisting}

In the packages section, clevis-dracut is installed, to then be used at post to communicate with a Tang server(s), subscribe to them and remove the temporary password.

\newpage
\subsection{Test Environment}
\begin{itemize}
  \item Two Tang servers using the tang puppet profile.
  \item A client with the clevis puppet profile.
  \item The client VM (clevis01.cp.lsst.org) is provisioned through PXE with 'Encrypted VDA' Partitioning Table and 'Kickstart Encrypted Provisioning Template'.
  \item During partition creation, clevis01 root partition is encrypted through LUKS with the passphrase 'temppass'.
  \item Then at packages, clevis-dracut is installed to then communicate with the Tang servers at post section.
  \item At post, clevis01 subscribes to the Tang servers (tang01.cp.lsst.org and tang02.cp.lsst.org) and the temporary passphrase encryption key is removed as a decryption mechanism.
  \item The provisioning continues and once concluded, a passphrase prompt is requested. This prompt holds on screen until the network interface is up and communicates to the Tang servers.
\end{itemize}

\newpage
\subsection{Lab Results}
\begin{itemize}
  \item The encrypted client clevis01 successfully decrypt during dracut by reaching tang01.
  \item The primary Tang server (tang01) was powered off and the client was able to decrypt through tang02.
  \item Both Tang servers were powered off and the server remains on hold requesting a passphrase (which doesn't exist) until at least one of the Tang servers is back online (see attach Image).
  \item For the scope of this PoC, the deletion and recreation of one or both Tang servers was not done, but presumably the client decryption would not happened and the content would be irrecoverable.
  \item One way of handling the loss of all Tang servers, is to add the keys to lsst-private repo, but key rotation is suggested by the documentation to increase safety.
\end{itemize}

\begin{figure}
  \includegraphics[width=16cm]{images/image1.png}
  \centering
  \caption{Access to LUKS encrypted drive while Tang server is rebooting}
\end{figure}

\newpage
\subsection{Conclusions - Pros and Cons}
\begin{itemize}
  \item The system disk MUST be encrypted during the provisioning, meaning that systems that have already been provisioned cannot be encrypted without risking data loss.
  \item Disk decryption successfully proved with the usage of Clevis - on the client side - and Tang - on the server side -. 
  \item During boot, it starts the network interface and contacts the Tang server for automated decryption.
  \item Since this occurs during provisioning, puppet cannot handle the root partition encryption, only if it was a non-root partition.
\end{itemize}


\newpage